\section{VMF Semantics}

\begin{figure}
\begin{grammar}
  Expr.     & $[[e]]$& $\bnfas$& $[[pe @: ae]]$ & Annotated pre-expression
  \\
  Pre-Expr. & $[[pe]]$ & $\bnfas$ & $[[force v]]$ & Unsuspend (force) thunk
  \\ &&& $\bnfaltbrk [[lam x. e]]$ & Function abstraction
  \\ &&& $\bnfaltbrk [[ e v ]]$ & Function application
  \\ &&& $\bnfaltbrk [[let x = e1 in e2]]$ & Bind computed value
  \\ &&& $\bnfaltbrk [[ret v]]$ & Produce a value
  \\ &&& $\bnfaltbrk [[ref v]]$ & Allocate store reference
  \\ &&& $\bnfaltbrk [[set v1 v2]]$ & Store mutation
  \\ &&& $\bnfaltbrk [[get v]]$ & Store projection
  \\ &&& $\bnfaltbrk [[ext v1 v2 v3]]$ & Dictionary extension
  \\ &&& $\bnfaltbrk [[proj oa v1 v2]]$ & Dictionary projection
  \\ &&& $\bnfaltbrk [[e ?: ae]]$ & Annotation Ascription
  \\[2px]
  Value    & $[[v]]$& $\bnfas$& $[[pv @: av]]$ & Annotated pre-value
  \\
  Pre-Val. & $[[pv]]$ 
  &$\bnfas$&        $[[othunk e]]$    & Open thunk
  \\ &&& $\bnfaltbrk [[thunk env e]]$ & Closed thunk
  \\ &&& $\bnfaltbrk [[dict delta]]$  & Dictionary
  \\ &&& $\bnfaltbrk [[num n]]$ & Number
  \\ &&& $\bnfaltbrk [[str s]]$ & String
  \\ &&& $\bnfaltbrk [[loc l]]$ & Store location
  \\ &&& $\bnfaltbrk [[()]]$    & Unit value
  \\ &&& $\bnfaltbrk [[x]]$     & Value variable
  \\
  Dict. & $[[ delta ]]$
  &$\bnfas$& $[[ emp ]]~|~[[ delta , v1 => v2 ]]$ & Dictionaries of values
  \\[2mm]
  Annot.
    & $[[oa]]$ & $\bnfas$ & $[[ ? ]]~|~[[ ! ]]$ & Uncertain vs. verified
  \\
    & $[[av]]$ & $\bnfas$ & $\cdots$ & Value annotation
  \\
    & $[[ae]]$ & $\bnfas$ & $\cdots$ & Expression annotation

\end{grammar}
\caption{Syntax of VMF Programs}
\label{fig-syntax}
\end{figure}

\subsection{VMF Syntax \& Dynamics}


\paragraph{VM State syntax.}

\begin{figure}
  \begin{grammar}
    State
    & $[[state]]$ & $\bnfas$ & $[[ <store; stack; env; pe> ]]$ &
    \\
    Store
    & $[[store]]$ & $\bnfas$ & $[[ emp ]]~|~[[ store, l => v ]]$ & Maps locations to values
    \\
    Stack
    & $[[stack]]$ & $\bnfas$ & $[[ halt ]]$ & Empty stack
    \\
    &&& $\bnfaltbrk [[ stack :::: (env, x.e)  ]]$ & Waiting for return
    \\
    &&& $\bnfaltbrk [[ stack :::: v  ]]$ & Fun. application argument
    \\
    Environment
    & $[[env]]$   & $\bnfas$ & $[[ emp ]]~|~[[ env,   x => v ]]$ & Maps variables to values
  \end{grammar}
  \caption{VM State Definitions}
  \label{fig:state}
\end{figure}


\paragraph{VM Dynamics.}


\begin{figure*}
\small
\[
\begin{array}{rclcl}
[[ <store; stack; env; let x = (pe1@:ae1) in e2> --|--> <store; stack :::: (env, x. e2); env; pe1> ]]
\\
[[ <store; stack :::: (env, x.(pe@:ae)); env'; ret v> --|--> <store; stack; env, x => v'; pe> ]]
& \textrm{when} & [[ env v ~> v' ]]
\\
[[ <store; stack; env; (pe@:ae) v> --|--> <store; stack :::: v'; env; pe> ]]
& \textrm{when} & [[ env v ~> v' ]]
\\
[[ <store; stack :::: v; env; lam x.(pe@:ae)> --|--> <store; stack; env, x => v; pe> ]]
\\
[[ <store; stack; env; force v> --|--> <store; stack; env'; pe> ]]
& \textrm{when} & [[ env v ~> (thunk env' (pe@:ae) @: av) ]]
\\
[[ <store; stack; env; ref v> --|--> <store, l => v; stack; env; ret (loc l @: ?)> ]]
& \textrm{when} & [[ l notin store ]]
\\
[[ <store; stack; env; set v1 v2> --|--> <store, l => v2'; stack; env; ret (() @: ?)> ]]
& \textrm{when} & [[ env v1 ~> (loc l @: av) ]]~\textrm{and}~[[ env v2 ~> v2' ]]
\\
[[ <store, l=>v2; stack; env; get v1> --|--> <store, l=>v2; stack; env; ret v2> ]]
& \textrm{when} & [[ env v1 ~> (loc l @: av) ]]
\\
[[ <store; stack; env; ext v1 v2 v3> --|--> <store; stack; env; ret (dict (delta, v2'=>v3')@:?)> ]]
& \textrm{when} & 
[[ env v1 ~> (dict delta @: av) ]],~[[ env v2 ~> v2' ]]~\textrm{and}~[[ env v3 ~> v3' ]]
\\
[[ <store; stack; env; proj ! v1 v2> --|--> <store; stack; env; ret v3> ]]
& \textrm{when} & [[ env v2 ~> v2' ]]~\textrm{and}~[[ env v1 ~> (dict (delta, v2' => v3) @: av) ]]
\end{array}
\]
\caption{Small-step, abstract machine semantics of VMF}
\end{figure*}

\clearpage

\section{Gradual Typing for Simple Databases}

\begin{figure}
  \begin{grammar}
  \end{grammar}
\end{figure}

\begin{figure}
  \begin{grammar}
  & $e$
    &$\bnfas$& $\cdots$ & Existing forms (\Figref{fig-syntax})
    \\ &&& $\bnfaltbrk [[ openDb oa v ]]$ & Open database by file path
    \\ &&& $\bnfaltbrk [[ filterDb oa v1 v2 ]]$ & Filter DB by predicate
    \\ &&& $\bnfaltbrk [[ joinDb oa v1 v2 v3 v4 ]]$ & Join DBs using keys' value
\end{grammar}
\caption{Database Library Forms}
\label{fig:db-syntax}
\end{figure}


\begin{figure}
\begin{grammar}
  Annotations
    & $[[av]]$ & $\bnfas$ & $[[A]]$ & Value annotation
  \\
    & $[[ae]]$ & $\bnfas$ & $[[C]]$ & Expression annotation
  \\
  Value Types  & $[[A]], [[B]]$& $\bnfas$& 
  $ [[U C]]$& Thunked computation
  \\ &&& $\bnfaltbrk [[Dict Delta]]$& Dictionary
  \\ &&& $\bnfaltbrk [[Num]]$& Number
  \\ &&& $\bnfaltbrk [[Str]]$& String
  \\ &&& $\bnfaltbrk [[Ref A]]$& Store reference
  \\ &&& $\bnfaltbrk [[1]]$& Unit
  \\ &&& $\bnfaltbrk [[?]]$& Unknown value type
  \\ &&& $\bnfaltbrk [[Db A]]$& Database; multiset of $[[A]]$s
  \\[2mm]
  Dictionary & $[[Delta]]$ & $\bnfas$ & \multicolumn{2}{l}{$[[emp]]~|~[[Delta, v => A]]$~~~\textrm{Maps values to types}}
  \\[2mm]
  Computation & $[[C]], [[D]]$ & $\bnfas$& 
  %\\ &&& 
  $[[A -> C]]$ & Function abstraction
  \\ 
  Types
  &&& $\bnfaltbrk [[F A]]$ & Value production
\end{grammar}
\caption{Type Syntax: Annotations for Values and Expressions}
\label{fig:types}
\end{figure}


\begin{figure}
\small
\begin{mathpar}
\Infer{annot}
{[[Ctx |- e <=  C]]}
{[[Ctx |- e ?: C => C ]]}
\and
\Infer{sub}
{[[Ctx |- e => C ]] 
  \\ 
  [[C <consis> D]]}
{[[Ctx |- e <= D ]]}
\and
\Infer{lam}
{[[Ctx, x : A |- e       <=      C]]}
{[[Ctx        |- lam x.e <= A -> C]] }
\and
\Infer{app}
{[[Ctx |- e => A -> C]]
\\\\
[[Ctx |- v <= A]]
}
{[[Ctx |- e v => C]]}
\and
\Infer{}
{[[Ctx |- v1 => ?]]
\\\\
[[Ctx |- v2 => ?]]
}
{[[Ctx |- proj ? v1 v2 => F ?]] }
\and
\Infer{}
{[[Ctx |- v1 => Dict (Delta, v2 => A)]]
\\\\
[[Ctx |- v2 => B]]
}
{[[Ctx |- proj ! v1 v2 => F A]] }
\and
\Infer{openDb?}
{[[Ctx |- v => Str]]}
{[[Ctx |- openDb ? v => F (Db ?)]]}
\and
\Infer{filterDb?}
{[[Ctx |- v1 => Db ?]]
\\
[[Ctx |- v2 <= U(? -> F 2)]]
}
{[[Ctx |- filterDb ? v1 v2 => F (Db ?)]]}
\and
\Infer{filterDb!}
{[[Ctx |- v1 => Db A]]
\\
[[Ctx |- v2 <= U(A -> F 2)]]
}
{[[Ctx |- filterDb ! v1 v2 => F (Db A)]]}
\and
\Infer{joinDb?}
{[[Ctx |- v1 => Db ?]]
\\
[[Ctx |- v2 => B2]]
\\\\
[[Ctx |- v3 => Db ?]]
\\
[[Ctx |- v4 => B4]]
}
{[[Ctx |- joinDb ? v1 v2 v3 v4 => F (Db ?)]]}
\and
\Infer{joinDb!}
{[[Ctx |- v1 => Db (Dict (Delta1, v2 => A))]]
\\
[[Ctx |- v2 => B2]]
\\
[[Ctx |- v3 => Db (Dict (Delta3, v4 => A))]]
\\
[[Ctx |- v4 => B4]]
}
{[[Ctx |- joinDb ! v1 v2 v3 v4 : F (Db (Dict (Delta1, v2 => A, Delta3, v4 => A)))]]}
\end{mathpar}
\caption{Selected typing rules}
\end{figure}

\paragraph{Library primitives: Syntax \& Dynamics.}

\clearpage
\appendix
\section{Appendix}


\paragraph{Program syntax: CBPV.}
For economy of both implementation and reasoning, the syntax of VMF
programs is directly insipired by that of the Call-by-Push-Value
(CPBV) syntax and evaluation strategy~\cite{CBPV}.
%
CBPV is similar to CBV, except that unlike CBV, CBPV neatly separates
the concept of a thunked computation from that of a lambda
abstraction.
%
Like other administrative normal forms, CPBV also imposes a separation
of values and computations, in terms of syntax and semantics.
%
Computations that eliminate values have value sub-terms.
%
In contrast to computation forms, value forms consist of (passive,
non-doing) data types; specifically, here we consider variables
(ranging over previously-defined values), primitives (viz., numbers
and strings), records of values, store locations, open thunks that
suspend a computation term as a value~($[[othunk e]]$) and thunks
closed by an environment~($[[thunk env e]]$).

Unlike values, computations consist of program terms that actively
\emph{do things}, and for which the VM defines operational steps. For
instance, field projection from a record value is a computation, since
it eliminates a value (viz., the record that contains the projected
field) and introduces a new one.
%
Though CBV treats lambdas as values, the distinctive quirk of CBPV is
that function abstraction is a computation (not a value) and that it
interacts with function applications via an ambient stack (hence,
``call-by-\emph{push}-value''):
%
Function application~$[[e v]]$ uses a function abstraction by pushing
an argument value~$v$ on this ambient stack and then running the
function body~$e$.
%
Lambdas~$[[lam x.e]]$ abstract over a variable~$x$ by popping a value
for~$x$ from this ambient stack.
%
Compared with CBV, the presence of this ambient stack streamlines the
expression of currying.
%
Specifically, CBPV avoids the need for creating closures that
represent the partially-applied lambdas in the curried series.
%
Below, we will augment these (standard) computations and values with
computations over simple databases.

\begin{figure}
\small
\begin{mathpar}
\Infer{lam}
{[[Ctx, x : A |- e       : C]]}
{[[Ctx        |- lam x.e : A -> C]] }
\and
\Infer{app}
{[[Ctx |- e : A -> C]]
\\\\
[[Ctx |- v : A]]
}
{[[Ctx |- e v : C]]}
\and
\Infer{proj?}
{[[Ctx |- v1 : ?]]
\\\\
[[Ctx |- v2 : ?]]
}
{[[Ctx |- proj ? v1 v2 : F ?]] }
\and
\Infer{proj!}
{[[Ctx |- v1 : Dict (Delta, v2 => A)]]
\\\\
[[Ctx |- v2 : B]]
}
{[[Ctx |- proj ! v1 v2 : F A]] }
\and
\Infer{openDb?}
{[[Ctx |- v : Str]]}
{[[Ctx |- openDb ? v : F (Db ?)]]}
\and
\Infer{filterDb?}
{[[Ctx |- v1 : Db ?]]
\\
[[Ctx |- v2 : U(? -> F 2)]]
}
{[[Ctx |- filterDb ? v1 v2 : F (Db ?)]]}
\and
\Infer{filterDb!}
{[[Ctx |- v1 : Db A]]
\\
[[Ctx |- v2 : U(A -> F 2)]]
}
{[[Ctx |- filterDb ! v1 v2 : F (Db A)]]}
\and
\Infer{joinDb?}
{[[Ctx |- v1 : Db ?]]
\\
[[Ctx |- v2 : B2]]
\\\\
[[Ctx |- v3 : Db ?]]
\\
[[Ctx |- v4 : B4]]
}
{[[Ctx |- joinDb ? v1 v2 v3 v4 : F (Db ?)]]}
\and
\Infer{joinDb!}
{[[Ctx |- v1 : Db (Dict (Delta1, v2 => A))]]
\\
[[Ctx |- v2 : B2]]
\\
[[Ctx |- v3 : Db (Dict (Delta3, v4 => A))]]
\\
[[Ctx |- v4 : B4]]
}
{[[Ctx |- joinDb ! v1 v2 v3 v4 : F (Db (Dict (Delta1, v2 => A, Delta3, v4 => A)))]]}
\end{mathpar}
\caption{Selected typing rules}
\end{figure}
